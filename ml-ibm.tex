\documentclass[9pt]{extarticle} 
\usepackage[utf8]{inputenc}
\usepackage[T1]{fontenc}
\usepackage{lmodern}          %
\usepackage{microtype}       
\usepackage[a4paper,margin=1in]{geometry}
\usepackage{amsmath, amssymb, amsfonts}
\usepackage{mathtools}        % More powerful math extensions
\usepackage{graphicx}
\usepackage{caption}
\usepackage{subcaption}
\usepackage{float}
\usepackage{wrapfig}
\usepackage{tikz}             % Drawing diagrams
\usepackage{booktabs}         % Nicer tabl
\usepackage{array}
\usepackage{multirow}
\usepackage{tabularx}
\usepackage{colortbl}
\usepackage[dvipsnames]{xcolor}
\usepackage[colorlinks=true, linkcolor=blue, citecolor=blue, urlcolor=blue]{hyperref}
\usepackage[nameinlink]{cleveref} % Smarter cross-referencing
\usepackage{enumitem}
\usepackage{fancyhdr}
\usepackage{hyperref}
\newcommand{\notimplies}{%
  \mathrel{{\ooalign{\hidewidth$\not\phantom{=}$\hidewidth\cr$\implies$}}}}
\usetikzlibrary{arrows.meta, positioning}
\pagestyle{fancy}
\fancyhf{}
\lhead{Machine Learning with Python}
\rhead{\today}
\cfoot{\thepage}
\newtheorem{definition}{Definition}[section]
\newtheorem{lemma}[definition]{Lemma}
\newtheorem{theorem}[definition]{Theorem}

\usepackage{tikz-cd}
\begin{document}
\begin{titlepage}
    Machine Learning with Python 
\end{titlepage}
\tableofcontents
\newpage
\section{Overview}

Machine learning is a subset of AI that uses 
computer algorithms that require feature engineering.  It 
teaches computers to learn from data and identify patterns and use 
them to make decisions without receiving explicit input 
from the user. There are three main types of machine learning models:
\begin{enumerate}
    \item Supervised learning models are trained on a known 
set of features and a target variable, to identify 
relationships which are then used for inference or 
forecasting on new data. For example: linear regression. 
    \item Unsupervised learning 
    models do not admit labelled feature-target style data, instead 
    they are trained on a set of variables which are all 
    considered features; the model finds relationships between these features. For 
    example: Principal Component Analysis.
    \item Reinforcement learning models simulate an AI agent interacting 
with its environment, they learn how to make decisions based on 
feedback from the environment. For example:
\end{enumerate}


The two focuses of supervised learning are regression and classification
One of the main types of unsupervised learning are clustering

\subsection{Machine Learning Lifecycle}

\begin{enumerate}
    \item \textbf{Problem Definition:} Clearly state the objective and desired outcome.
    
    \item \textbf{Data Collection:} Identify required data and its source.
    
    \item \textbf{Data Preparation:} Clean the data, handle missing values, normalize if necessary, engineer features, and perform exploratory data analysis. Split into training and testing sets.
    
    \item \textbf{Model Development:} Train the model, tune hyperparameters, and evaluate performance using appropriate metrics.
    
    \item \textbf{Deployment:} Integrate the trained model into a production environment.
\end{enumerate}


\subsection{Difference between a Data Scientist and AI Engineer} 

\begin{table}[h!]
\centering
\begin{tabular}{|p{3.5cm}|p{5cm}|p{5cm}|}
\hline
\textbf{Aspect} & \textbf{Data Science} & \textbf{AI Engineering} \\
\hline
\textbf{Primary Use Cases} & Descriptive and predictive analytics (e.g., EDA, clustering, regression, classification) & Prescriptive and generative AI (e.g., optimisation, recommendation systems, intelligent assistants) \\
\hline
\textbf{Data Type Focus} & Primarily structured/tabular data, cleaned and preprocessed & Primarily unstructured data (text, images, audio, video), used at large scale \\
\hline
\textbf{Model Characteristics} & Narrow-scope ML models, smaller in size, domain-specific, faster to train & Foundation models, large-scale, general-purpose, high compute and data requirements \\
\hline
\textbf{Development Process} & Data-driven model development (feature engineering, training, validation) & Application of pre-trained models with prompt engineering, PEFT, and RAG frameworks \\
\hline
\end{tabular}
\caption{Key Differences Between Data Science and AI Engineering}
\end{table}


\subsection{Tools for Machine Learning}

Python has several modules that handle the different stages of the 
machine learning model development pipeline:
\begin{enumerate}
    \item Data preprocessing: \texttt{pysql}, \texttt{pandas}
    \item Exploratory data analysis: \texttt{pandas}, \texttt{numpy}, \texttt{matplotlib}
    \item Optimisation: \texttt{scipy}
    \item Implementation: \texttt{scikit-learn} (supervised and unsupervised methods), 
    \texttt{keras}, \texttt{pytorch} (deep learing)
\end{enumerate}

\subsection{Scikit-learn}

The basic syntax for using supervised learning models in \texttt{scikit-learn} follows a standard workflow:

\begin{enumerate}
    \item \textbf{Split the data:}
    \begin{verbatim}
x_train, X_test, y_train, y_test = train_test_split(X, y, test_size=...)
    \end{verbatim}

    \item \textbf{Import and initialise the model:}
    \begin{verbatim}
from sklearn import svm
model = svm.SVC(...)
    \end{verbatim}

    \item \textbf{Train the model:}
    \begin{verbatim}
model.fit(X_train, y_train)
    \end{verbatim}

    \item \textbf{Make predictions:}
    \begin{verbatim}
predictions = model.predict(X_test)
    \end{verbatim}

    \item \textbf{Evaluate performance:}
    Use a confusion matrix to assess classification accuracy:
    \begin{verbatim}
from sklearn.metrics import confusion_matrix
confusion_matrix(y_test, predictions)
    \end{verbatim}

    \item \textbf{Optional – Save the model:}
    You can serialise the trained model using \texttt{pickle}:
    \begin{verbatim}
import pickle
with open("model.pkl", "wb") as f:
    pickle.dump(model, f)
    \end{verbatim}
    Whether this step is necessary depends on the context and industry practice.
\end{enumerate}

\section{Linear Regression}

Regression is a type of supervised learning model. It models a relationship between 
a continuous target variable and explanatory features.
\subsection{Introduction to Regression}


\subsubsection*{Definition}

Linear regression models the relationship between a response variable $Y$ and one or more features $X_1, \dots, X_p$. In the case of simple linear regression with one predictor $X$, the model assumes:

\[
Y \approx \beta_0 + \beta_1 X
\]

where $\beta_0$ is the intercept and $\beta_1$ is the slope.

\subsubsection*{Coefficient Estimation}

Given training data $(x_1, y_1), \dots, (x_n, y_n)$, the goal is to estimate the coefficients $\beta_0, \beta_1$ such that the fitted values are as close as possible to the observed values. This is done by minimizing a norm of the residual vector:

\[
\beta_{\text{min}} = \arg\min_{\beta \in \mathbb{R}^2} \| \mathbf{Y} - \mathbf{X} \beta \|_p^p
\]

For ordinary least squares (OLS), we use the $2$-norm ($p=2$), 
leading to the minimisation of the residual sum of squares (RSS). 
The OLS solution yields closed-form expressions for $\hat{\beta}_0$ and 
$\hat{\beta}_1$.
 For least absolute deviations (LAD), we take $p = 1$. 
 For Chebyshev or minimax regression, we take $p = \infty$, minimising the maximum residual.


\subsubsection*{Population vs Sample Regression}

In the real world we almost always do not have a maximal data set. 
This means that we do not have data for every single item in the population, 
all we can hope to obtain are samples. 
The fitted regression line from a 
sample \textit{estimates} the \emph{population regression line}:

\[
Y = \beta_0 + \beta_1 X + \varepsilon
\]


where $\varepsilon$ is the irreducible error due to unobserved factors. Since we cannot observe the full population, we fit the model on a sample and obtain estimates $\hat{\beta}_0, \hat{\beta}_1$.
However, since we only observe a sample, the estimates $\hat{\beta}$ vary
from sample to sample\footnote{
 The goal of statistical inference is to understand 
how close $\hat{\beta}$ is to the true $\beta$ and quantify this uncertainty.}.
The sample regression is thus an estimate of the population regression. 
If we had access to all population data, the OLS minimisation would yield the true $\beta_0$, $\beta_1$.


\subsubsection*{Sampling and the Central Limit Theorem}

Across repeated samples of size $n$, we obtain different estimates of $\beta$, say $\hat{\beta}^{(1)}, \dots, \hat{\beta}^{(n)}$. Their average converges to the true coefficient $\bar{\beta}$ as $n \to \infty$, due to the central limit theorem:

\[
\hat{\beta} - \bar{\beta} \xrightarrow{d} \mathcal{N}(0, \sigma_*)
\]

\subsubsection*{Standard Error and Inference}

The variance of the coefficient estimates across samples is the \emph{standard error}:

$$SE(\hat{\beta}_i) = f(\sigma^2, x_j, \bar{x})$$

where $\sigma^2$ is estimated from the data via:

\[
\hat{\sigma}^2 = \frac{RSS}{n - 2}
\]
The standard error quantifies the variability of the estimated coefficient 
under repeated sampling. If we observe a high standard error then it means that we do not see 
replicability of the relationship as we vary our sample, which may suggest that 
the relationship between the features and target is ephemeral and noisy, so 
any forecast or inference made using it will be unreliable.

\end{document}